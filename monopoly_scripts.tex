
\begin{description}

% \item[Geogebra] 

% \link{  .ggb}{GeoGebra applet}

\item[R] 

\link{http://www.math.unl.edu/~sdunbar1/monopoly40.R}{R script for
  40-state Monopoly.}

\begin{lstlisting}[language=R]
library("markovchain")

rotvec <- function(vec) vec[ c(length(vec), 1:(length(vec)-1)) ]

nStates <- 40
monopolyStates <- as.character( c(1:nStates) )

diceRoll <- matrix(0, nStates, nStates)
p1 <- c(0, 0,
        1/36, 2/36, 3/36, 4/36, 5/36, 6/36, 5/36, 4/36, 3/36, 2/36, 1/36,
        numeric(nStates-13)
        )
for (i in 1:nStates) {
    diceRoll[i, ] <- p1
    p1 <- rotvec(p1)
}

Jail <- diag(nStates)
Jail[30,30] <- 0
Jail[30, 10] <- 1

Ch <- diag(nStates)
chanceStates <- c(7, 22, 36)
chanceDests <- c(5, 10, 11, 24, 39, 40)
Ch[ cbind( chanceStates, chanceStates ) ] <- 7/16
Ch[ cbind( chanceStates, chanceStates-3 ) ] <- 1/16 #Go Back 3 spaces
Ch[ chanceStates, chanceDests] <- 1/16
Ch[7,12] <- 1/16; Ch[22,28] <- 1/16; Ch[36, 12] <- 1/16 #nearest utility
Ch[7,15] <- 1/16; Ch[22,25] <- 1/16; #nearest Railroad,
Ch[36, 5] <- 2/16 # note 2 ways to get to Reading Railroad

Cc <- diag(nStates)
chestStates <- c(2, 17, 33)
chestDests <-  c(10, 40)
Cc[ cbind(chestStates, chestStates) ] <- 14/16
Cc[chestStates, chestDests] <- 1/16

monopolyMatrix <- diceRoll %*% Jail %*% Ch %*% Cc

mcMonopoly <- new("markovchain", states = monopolyStates, byrow = TRUE,
                  transitionMatrix = monopolyMatrix, name = "Monopoly")
ss <- steadyStates(mcMonopoly)

top14ofAbbott <- (1/100)*c(5.896, 3.188, 3.114, 3.071, 3.064, 2.993, 2.919, 2.917, 2.875, 2.830, 2.811, 2.777, 2.739, 2.736)
top14ofAbbottIndices <- c(10, 24, 40, 19, 25, 5, 15, 18, 20, 21, 28, 16, 23, 11)

sstop14 <- sort(ss, decreasing=TRUE)[1:14]
sstop14Indices <- order(ss, decreasing=TRUE)[1:14]

maxDiff <- max( abs( sstop14 - top14ofAbbott) )
cat("Maximum Difference: ", maxDiff, "\n")
maxDiffIndex <- which.max( abs( sstop14 - top14ofAbbott) )
cat("Maximum Difference Index: ", maxDiffIndex, "\n")
maxRelDiff <- maxDiff/sstop14[maxDiffIndex]
cat("Relative Maximum Difference: ", maxRelDiff, "\n")

ssdf <- data.frame( sstop14Indices, sstop14)

barplot(ss, names.arg=1:40,
        main="Steady State Distribution for 40-State Monopoly",
        xlab="State", ylab="Probability")
\end{lstlisting}

\link{http://www.math.unl.edu/~sdunbar1/monopoly42.R}{R script for
  42-state Monopoly.}

\begin{lstlisting}
  library("markovchain")

rotvec <- function(vec) vec[ c(length(vec), 1:(length(vec)-1)) ]

nStates <- 42
monopolyStates <- as.character( c(1:nStates) )

diceRoll <- matrix(0, (nStates-2), (nStates-2))
p1 <- c(0, 0,
        1/36, 2/36, 3/36, 4/36, 5/36, 6/36, 5/36, 4/36, 3/36, 2/36, 1/36,
        numeric((nStates-2)-13)
        )
for (i in 1:(nStates-2)) {                #regular states 1 through 40
    diceRoll[i, ] <- p1
    p1 <- rotvec(p1)
}
## Next, no probability of going from board state to extra Jail states 
diceRoll  <- cbind(diceRoll, matrix(0, (nStates-2), 2))
## add rows from extra Jail states
diceRoll  <- rbind(diceRoll, matrix(0, 2, nStates))
## fill transitions for rolling doubles to get out of Jail
diceRoll[41, 41] <- 0; diceRoll[41, 42] <- 5/6 #stay in jail one turn
diceRoll[41, c(12, 14, 16, 18, 20, 22)] <- 1/36 #roll doubles
diceRoll[42, 42] <- 0; diceRoll[42, 10] <- 5/6 #stay in Jail one more turn
diceRoll[42, c(12, 14, 16, 18, 20, 22)] <- 1/36 #roll doubles 

Jail <- diag(nStates)
Jail[30,30] <- 0
Jail[30, 41] <- 1                       #go from Police to Jail

Cc <- diag(nStates)
chestStates <- c(2, 17, 33)
chestDests <-  c(40, 41)
Cc[ cbind(chestStates, chestStates) ] <- 14/16
Cc[chestStates, chestDests] <- 1/16

Ch <- diag(nStates)
chanceStates <- c(7, 22, 36)
chanceDests <- c(5, 11, 24, 39, 40, 41)
Ch[ cbind( chanceStates, chanceStates ) ] <- 7/16
Ch[ cbind( chanceStates, chanceStates-3 ) ] <- 1/16 #Go Back 3 spaces
Ch[ chanceStates, chanceDests] <- 1/16
Ch[7,12] <- 1/16; Ch[22,28] <- 1/16; Ch[36, 12] <- 1/16 #nearest utility
Ch[7,15] <- 1/16; Ch[22,25] <- 1/16; #nearest Railroad,
Ch[36, 5] <- 2/16 # note 2 ways to get to Reading Railroad

monopMat <- diceRoll %*% Jail %*% Ch %*% Cc

mcMonopoly42 <- new("markovchain", states = monopolyStates, byrow = TRUE,
                  transitionMatrix = monopMat, name = "Monopoly")
ss <- steadyStates(mcMonopoly42)

ssComb  <- ss
ssComb[10]  <- ssComb[10] + ssComb[41] + ssComb[42]
ssCombTop14 <- sort(ssComb[1:40], decreasing=TRUE)[1:14]
ssCombTop14Indices <- order(ssComb[1:40], decreasing=TRUE)[1:14]
ssCombdf <- data.frame( ssCombTop14Indices, ssCombTop14)

top14ofAbbott <- (1/100)*c(10.800, 3.013, 2.949, 2.905, 2.845, 2.822, 2.821, 2.674, 2.669, 2.659, 2.659, 2.636, 2.628, 2.603)
top14ofAbbottIndices <- c(10, 24, 40, 25, 5, 20, 19, 18, 16, 28, 15, 12, 21 , 11)

maxDiff <- max( abs( ssCombTop14 - top14ofAbbott) )
cat("Maximum Difference: ", maxDiff, "\n")
maxDiffIndex <- which.max( abs( ssCombTop14 - top14ofAbbott) )
cat("Maximum Difference Index: ", maxDiffIndex, "\n")
maxRelDiff <- maxDiff/ssCombTop14[maxDiffIndex]
cat("Relative Maximum Difference: ", maxRelDiff, "\n")

barplot(ssComb[1:40], names.arg=1:40,
        main="Steady State Distribution for 42-State Monopoly, 10,41,42 Combined",
        xlab="State", ylab="Probability")
\end{lstlisting}

\link{http://www.math.unl.edu/~sdunbar1/simulateMonopoly42.R}{R script to
  simulate 42-state Monopoly.}

\begin{lstlisting}
  ##  This function calculates where you will land on your next move.
moveSquare <- function(current, jailTime) {
    dice <- sample(seq(1,6), 2, replace=TRUE)

    if (current == 10 & jailTime > 0) { # in Jail
        if (dice[1] == dice[2]) {           # roll doubles!
            current <- 10 + sum(dice)       # get out of Jail
            jailTime <- 0
        } else {                            #don't move but...
            jailTime <- (jailTime + 1) %% 3 # count up jail time
        } 
    } else {
        current <- current + sum(dice)      # regular roll
    }
  return(c(current, jailTime))
}

## Helper function to avoid code replication
checkState <- function(current) {
  if (current > 40) {
    current <- current - 40
  } else if (current < 1) {
    current <- current + 40
  }
  return(current)
}

## Helper function to record landings, 
updateStateVector <- function(current, jailTime, landings) {
    if (current == 10 & jailTime > 0) {
        landings[40 + jailTime] = landings[40 + jailTime] + 1
    } else {
        landings[current] <- landings[current] + 1
    }
    
    return(landings)
}

## Move according to community chest cards
## Use 16 cards, Get Out of Jail is reserved
communityChest <- function(current, jailTime) {
  u <- runif(1)
  goto <- current # Default. Do nothing
  if (u < 1 / 16) {
    goto <- 40 # Go
  } else if (u < 2 / 16) {
      goto <- 10 # Jail
      jailTime <- 1
  }
  return( c(goto, jailTime) )
}

## Move according to chance cards
## Use 16 cards, Get Out of Jail is NOT reserved
chance <- function(current, jailTime) {
  u <- runif(1)
  goto <- current # Default. Do nothing
  if (u < 1 / 16) {
      goto <- 40 # Go
  } else if (u < 2 / 16) {
      goto <- 24 # Illinois Ave.
  } else if (u < 3 / 16) {
      goto <- 11 # St. Charles Place
  } else if (u < 4 / 16) {
      goto <- 10 # Jail
      jailTime <- 1
  } else if (u < 5 /16 ) {
      goto <- 5 #Take ride on Reading
  } else if (u < 6 / 16) {
      goto <- 39 # Boardwalk
  } else if (u < 7 / 16) {
      goto <- checkState(current - 3) # Must check, since goto maybe negative!
  } else if (u < 8 / 16) {
      if (current > 28  | current < 12) { # Advance to nearest Utility
      goto <- 12 # Water Works
    } else {
      goto <- 28 # Electricity
    }
  } else if (u < 9 / 16) { # Advance to nearest Railroad
    if (current > 35  | current < 5) { 
      goto <- 5                         # Reading
    } else if (current > 5  & current < 15) {
      goto <- 15 #Pennsylvania
    } else if (current > 15  & current < 25) {
      goto <- 25 # B & O
    } else if (current > 25  & current < 35) {
      goto <- 35 # Short Line
    }
}
  return(c(goto, jailTime))
}

## Main, record where landings occur
mySimulateMonopoly <- function (numberOfTurns) 
{
    landings <- numeric(42) # vector to hold occupancy of states,
    # 1 is Med Ave, 40 is Go
    # 41 is in Jail for one turn, 42 is Jail for two turns 
    # 10 is Just Visiting Jail or Jail for three turns, ready to get out
    jailTime <- 0
    current <- 40           # start at Go
    for (i in 1:numberOfTurns) {
        moveState <- moveSquare(current, jailTime)
        current <- moveState[1]
        jailTime <- moveState[2]
        current <- checkState(current)
        if (current == 30) {
            current <- 10
            jailTime <- 1
        }
        if (current == 7 | current == 22 | current == 36) {
            move <- chance(current, jailTime)
            current <- move[1]
            jailTime <- move[2]
        }
        if (current == 2 | current == 17 | current == 33) {
            move <- communityChest(current, jailTime)
            current <- move[1]
            jailTime <- move[2]
        }
        landings <- updateStateVector(current, jailTime, landings) # must do update
    }
return(landings)
}

n  <- 1000
ss  <- mySimulateMonopoly(n)/n
\end{lstlisting}
%\item[Octave]

% \link{http://www.math.unl.edu/~sdunbar1/    .m}{Octave script for .}

% \begin{lstlisting}[language=Octave]

% \end{lstlisting}

% \item[Perl] 

% \link{http://www.math.unl.edu/~sdunbar1/    .pl}{Perl PDL script for .}

% \begin{lstlisting}[language=Perl]

% \end{lstlisting}

% \item[SciPy] 

% \link{http://www.math.unl.edu/~sdunbar1/    .py}{Scientific Python script for .}

% \begin{lstlisting}[language=Python]

% \end{lstlisting}

\end{description}



%%% Local Variables:
%%% mode: latex
%%% TeX-master: t
%%% End:
