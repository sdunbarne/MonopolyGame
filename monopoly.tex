%%% -*-LaTeX-*-
%%% monopoly.tex.orig
%%% Prettyprinted by texpretty lex version 0.02 [21-May-2001]
%%% on Thu Jan 28 08:09:24 2021
%%% for Steven R. Dunbar (sdunbar@family-desktop)

\documentclass[12pt]{article}

\input{../../../../etc/macros}
%% \input{../../../../etc/mzlatex_macros}
\input{../../../../etc/pdf_macros}

\bibliographystyle{plain}

\begin{document}

\myheader \mytitle

\hr

\sectiontitle{The Game of Monopoly as a Markov Chain}

\hr

\usefirefox

\hr

% \visual{Study Tip}{../../../../CommonInformation/Lessons/studytip.png}
% \section*{Study Tip}

% \hr

\visual{Rating}{../../../../CommonInformation/Lessons/rating.png}
\section*{Rating} %one of
% Everyone: contains no mathematics.
% Student: contains scenes of mild algebra or calculus that may require guidance.
Mathematically Mature:  may contain mathematics beyond calculus with
proofs.  % Mathematicians Only: prolonged scenes of intense rigor.

\hr

\visual{Section Starter Question}{../../../../CommonInformation/Lessons/question_mark.png}
\section*{Section Starter Question}

While playing the game of Monopoly, how often do you ``take a walk on
the Boardwalk''?  That is, what is the fraction of time spent on the
most expensive property in the game?  How would you estimate this
fraction?

\hr

\visual{Key Concepts}{../../../../CommonInformation/Lessons/keyconcepts.png}
\section*{Key Concepts}

\begin{enumerate}
    \item
        Modeling the game of Monopoly as a Markov chain finds the
        stationary distribution to find the properties landed on most
        frequently.
    \item
        The modeling proceeds as a sequence of increasingly detailed
        Markov chains, each modeling more of the rules of the game.
    \item
        The transition probability matrix is composed as a product of
        simpler transition matrices, each modeling a specific type of
        moves around the board.
\end{enumerate}

\hr

\visual{Vocabulary}{../../../../CommonInformation/Lessons/vocabulary.png}
\section*{Vocabulary}
\begin{enumerate}
    \item
        A \defn{circulant matrix} is a square matrix in which each row
        vector is rotated one element to the right relative to the
        preceding row vector.
\end{enumerate}

\hr

\visual{Mathematical Ideas}{../../../../CommonInformation/Lessons/mathematicalideas.png}
\section*{Mathematical Ideas}
\subsection*{Monopoly as a Markov Chain }

This section assumes familiarity with the standard layout and rules of
the board game Monopoly, in particular, the classic U.S.\ version based
on Atlantic City NJ.\@ The goal is to model the game as a Markov chain
to find the stationary distribution to determine the real estate squares
a player token lands on most frequently.  Those squares on the board are
``more valuable'' in the sense that other players will land on them more
frequently and pay rent more frequently.  Deciding which color group
squares are ``best'' is more complicated because of the variable costs
of rent and improvement costs around the board and that aspect is not
considered here.  The main intention of this section is to show how to
successively make changes to a simple Markov chain to simulate a more
nuanced model incorporating more transitions among states.  Another
intention is to analyze a Markov chain larger than introductory Markov
chains.

The basic transition probability matrix around the board is simply
determined by the rolls of two fair six-sided dice.  However making
distant jumps due to the ``Go To Jail'' square and the drawing of
certain \emph{Community Chest} and \emph{Chance} cards complicate the
transition probability matrix.  To understand this complication, first
consider a much smaller example with just \( 4 \) spaces:  Jail is state
\( 1 \), Policeman is state \( 2 \), Community Chest is state \( 3 \),
and Go is state \( 4 \), see Figure~%
\ref{fig:monopoly:smallmonopoly}.  A fair two-sided ``die'' (i.e.\ a
fair coin) determines the \( 1 \) or \( 2 \) step forward motion of a
player token around the board.  Then the ``dice roll'' transition matrix
\( R \) is
\[
    R =
    \begin{pmatrix}
        0 & \frac{1}{2} & \frac{1}{2} & 0 \\
        0 & 0 & \frac{1}{2} & \frac{1}{2} \\
        \frac{1}{2} & 0 & 0 & \frac{1}{2}\\
        \frac{1}{2} & \frac{1}{2} & 0 & 0 \\
    \end{pmatrix}
    .
\] (The source of this section,~%
\cite{abbott97}, uses the alternative convention that \( P_{ij} \)
represents the probability of moving \emph{to} state \( i \) \emph{from}
state \( j \).  Thus all matrices here are the transpose of matrices in~%
\cite{abbott97}.) This transition probability matrix is regular, and
that the stationary distribution is \( (1/4, 1/4, 1/4, 1/4) \).

\begin{figure}
    \centering
    \setlength{\unitlength}{1cm}
    \begin{picture}(10,10)
        \put(0,0){\framebox(5,5){ \parbox{5\unitlength}{%
        \center{JAIL \\
        State 1}}}}
        \put(0,5){\framebox(5,5){ \parbox{5\unitlength}{%
        \center{POLICEMAN \\
        State 2}}}}
        \put(5,5){\framebox(5,5){ \parbox{5\unitlength}{%
        \center{COMMUNITY CHEST \\
        State 3}}}}
        \put(5,0){\framebox(5,5){ \parbox{5\unitlength}{%
        \center{GO \\
        State 4}}}}
    \end{picture}
    \caption{A Monopoly board with $ 4 $ spaces.}~%
    \label{fig:monopoly:smallmonopoly}
\end{figure}

The ``Jail'' matrix represents the transition from landing on the
Policeman:
\[
    J =
    \begin{pmatrix}
        1 & 0 & 0 & 0 \\
        1 & 0 & 0 & 0 \\
        0 & 0 & 1 & 0 \\
        0 & 0 & 0 & 1
    \end{pmatrix}
    .
\] This matrix differs from the identity matrix only in the second row (Policeman)
where, with probability \( 1 \), a player goes directly to state \( 1 \),
Jail.

The last element is the Community Chest space.  Of the \( 16 \)
Community Chest cards (including the Get Out of Jail Free card), \( 14 \)
penalize or reward some peculiar accomplishment, e.g.\ \( \$10 \) for
second prize in a beauty contest.  However, one card sends the player to
Jail, i.e.\ state \( 1 \), and the other card advances the player to Go,
state \( 4 \).  To form the Community Chest matrix, again start with the
identity matrix and edit the third row to get
\[
    C =
    \begin{pmatrix}
        1 & 0 & 0 & 0 \\
        0 & 1 & 0 & 0 \\
        \frac{1}{16} & 0 & \frac{14}{16} & \frac{1}{16} \\
        0 & 0 & 0 & 1
    \end{pmatrix}
    .
\] The transition matrix for the simplified \( 4 \)-space board is then
computed by the product
\[
    RJC =
    \begin{pmatrix}
        \frac{17}{32} & 0 & \frac{7}{16} & \frac{1}{32} \\
        \frac{1}{32} & 0 & \frac{7}{16} & \frac{17}{32} \\
        \frac{1}{2} & 0 & 0 & \frac{1}{2} \\
        1 & 0 & 0 & 0
    \end{pmatrix}
    .
\] The order of the factors reflects the process of the play:  A player
rolls first and then follows the actions of where the token lands, i.e.\
%
go to Jail, or draw a Community Chest card.  Now the second row of the
matrix gives the transition probabilities for a token on the Policeman's
square.  However, this square is never occupied, the second column of
the product is entirely \( 0 \).  Note that the conditional
probabilities of going from Policeman state \( 2 \) to other states are
not \( 0 \). In fact, those conditional probabilities are exactly what
would be expected starting from square \( 2 \).  But the probability of
being at the Policeman state is \( 0 \), so these correctly calculated
probabilities are irrelevant.  In terms of Markov chain states, the Go,
Jail, and Community Chest states are communicating and the Policeman
state is not accessible from the other states.  Deleting the second
column and row from the matrix gives the transitions between the
remaining \( 3 \) occupiable states.  The stationary distribution is \(
(\frac{16}{27}, \frac{7}{27}, \frac{4}{27}) \).  The high frequency of
landing in Jail in this truncated game is not surprising.

The states of the Markov chain representing the full game are the \( 40 \)
spaces around the board, numbered \( 1 \) through \( 40 \) with
Mediterranean Avenue as \( 1 \) and Go as \( 40 \).

The \( 40 \times 40 \) Rolling matrix has as its first row
\[
    [0, 0, \frac{1}{36}, \frac{2}{36},\frac{3}{36},\frac{4}{36},\frac{5}
    {36},\frac{6}{36}, \frac{5}{36},\frac{4}{36},\frac{3}{36},\frac{2}{36},\frac
    {1}{36}, 0, 0, \dots, 0].
\] The second row is a rotated version of this row, add a zero at the
front and shift the nonzero entries one column to the right.  Form each
succeeding row the same way until eventually the first entry is nonzero,
as the rolls start to wrap around the board.  This creates a \defn{circulant
matrix}%
\index{circulant matrix}%
, a square matrix in which each row vector is rotated one element to the
right relative to the preceding row vector.  Considering just this
transition probability matrix, the steady state distribution is
uniformly \( \frac{1}{40} = 0.025 \) which is expected.

For now, continue to ignore the more complicated rules about exiting
from the Jail square, and let the \( J \) matrix be just the identity
matrix with the \( 1 \) in row \( 30 \) column, Policeman, moved from
column \( 30 \) to column \( 10 \), Jail.

The Community Chest spaces or states are \( 2 \), \( 17 \), and \( 33 \).
Community Chest cards are most likely to give you money leaving the
state unchanged.  But two of the cards move the player either to Jail,
state \( 10 \) or to Go, state \( 40 \).  To form the Community Chest
matrix, \( \mathit{Cc} \) change the \( 1 \)s on the diagonal in row \(
30 \) of the identity matrix to \( \frac{14}{16} \) and enter \( \frac{1}
{16} \) in each of column \( 10 \), Jail, and column \( 40 \), Go.

The Chance spaces or states are \( 7 \), \( 22 \) and \( 36 \).  A
Chance card is more likely than a Community Chest card to move players, \(
10 \) of the \( 16 \) cards directing a player to a different state.
The existence of cards such as ``Go back three spaces'' and ``Advance to
the nearest railroad'' means the transitions depend on which Chance a
player lands on.  The Chance matrix \( \mathit{Ch} \) differs from the
identity in rows \( 7 \), \( 22 \) and \( 36 \).  It makes sense to
multiply the Community Chest matrix \emph{after} the Chance matrix,
since there is a probability \( \frac{1}{16} \) of moving from Chance
state \( 36 \) to Community Chest state \( 33 \) and thence to Go or
Jail, each with probability \( \frac{1}{16} \).

Finally, the full transition probability matrix is \( P = R J \mathit{Ch}
\mathit{Cc} \).  The steady state distribution gives a preliminary look
at the proportion of time a player spends in each state over the course
of a long game.  The steady state distribution as a bar graph is in
Figure~%
\ref{fig:monopoly:monopoly40}.  Table~%
\ref{tab:monopoly:freq40} summarizes the results with the \( 14 \) most
frequently visited states in this model.  The R script to generate the
first two columns of Table~%
\ref{tab:monopoly:freq40} from the steady state distribution (eigenvector)
of the transition probability matrix is in the Scripts section below.
The last two columns come from a simulation with \( 10^6 \) steps. All
states except \( 30 \) are communicating and the Policeman state \( 30 \)
is not accessible from the other states.  The frequencies computed here
are essentially the same as the corresponding table in~%
\cite{abbott97}, with differences due to the number of significant
digits reported. Furthermore, the decreasing order of eigenvalue
frequencies differs from the order of simulated frequencies in only \( 2
\) states, with a transpositions of states \( 21 \) and \( 28 \).  This
gives confidence to the eigenvector and the simulation values and also
indicates the number of steps necessary to get good agreement with the
theoretical values.

\begin{table}
    \centering
    \begin{tabular}{rrrrrrr}
           &       &            & Abbott & Abbott    & Simulation & Simulation \\ 
           & State & Frequency  & State  & Frequency & State      &            \\ 
        1  & 10    & 0.05896420 & 10     & 0.05896   & 10         & 0.058909   \\ 
        2  & 24    & 0.03187795 & 24     & 0.03188   & 24         & 0.031934   \\ 
        3  & 40    & 0.03113803 & 40     & 0.03114   & 40         & 0.031183   \\ 
        4  & 19    & 0.03071024 & 19     & 0.03071   & 19         & 0.030969   \\ 
        5  & 25    & 0.03063697 & 25     & 0.03064   & 25         & 0.030858   \\ 
        6  & 5     & 0.02993127 & 5      & 0.02993   & 5          & 0.029701   \\ 
        7  & 15    & 0.02918612 & 15     & 0.02919   & 15         & 0.029198   \\ 
        8  & 18    & 0.02916517 & 18     & 0.02917   & 18         & 0.029070   \\ 
        9  & 20    & 0.02874826 & 20     & 0.02875   & 20         & 0.028704   \\ 
        10 & 21    & 0.02830354 & 21     & 0.02830   & 28         & 0.028230   \\ 
        11 & 28    & 0.02810737 & 28     & 0.02811   & 21         & 0.028152   \\ 
        12 & 16    & 0.02776751 & 16     & 0.02777   & 16         & 0.028065   \\ 
        13 & 23    & 0.02738577 & 23     & 0.02739   & 23         & 0.027349   \\ 
        14 & 11    & 0.02735991 & 11     & 0.02736   & 11         & 0.027226   \\ 
    \end{tabular}
    \caption{The $ 14 $ most-frequently visited states in the simple
    $ 40 $-state model.}~%
    \label{tab:monopoly:freq40}
\end{table}

\begin{figure}
    \centering
    \includegraphics[scale=0.50]{monopoly40}
    \caption{Steady state distribution for the first $ 40 $-state
    model of Monopoly.}~%
    \label{fig:monopoly:monopoly40}
\end{figure}

\subsection*{Modeling Rules for Jail}

This first model and simulation ignores two finer aspects of the game.
This model assumes a player sent to Jail immediately pays the \( \$50 \)
fine or uses a Get Out of Jail card to exit Jail on the next roll.  But
late in a game, when the board is covered with hotels and houses with
high rent, players may choose to stay in Jail for a couple of turns to
avoid paying rents.  The rules allow this for two turns unless doubles
are rolled, in which case a player must leave, exposed to nearby high
rents.  Because one must leave Jail on the third roll, the model now
needs a record of how many unsuccessful rolls of doubles have occurred.
The solution is to define two new Jail states, one for being newly
Jailed, now state \( 41 \) and the other for having failed at doubles
exactly once, state \( 42 \).  This is a small example of a general
principle of turning a process which is not Markov into a Markov chain
by adding states, for a general discussion, see
\cite{Stroock2016}. The original Jail state \( 10 \) serves as both the
Just-Visiting and twice-failed-rolling-doubles states, since in both
cases, the player leaves on the next turn.

Table~%
\ref{tab:monopoly:freq42} gives the results of one simulation with \( 10^6
\) steps of this expanded Markov chain with the Jail states \( 10 \), \(
41 \) and \( 42 \) combined.  State \( 10 \) now is a state reached by
either a roll of the dice or by aging out of Jail so a player can move
on to other states. The steady state distribution of this expanded
Markov chain as a bar graph is in Figure~%
\ref{fig:monopoly:monopoly42}, also with the Jail states \( 10 \), \( 41
\), and \( 42 \) combined.

\begin{table}
    \centering
    \begin{tabular}{rrrrrrr}
           &            &            & Abbott     & Abbott    & Simulation & Simulation \\ 
           & State      & Frequency  & State      & Frequency & State      &            \\ 
        1  & Jail       & 0.10799193 & Jail       & 0.10800   & Jail       & 0.108605   \\ 
        2  & 24         & 0.03012586 & 24         & 0.03013   & 24         & 0.030191   \\ 
        3  & 40         & 0.02948697 & 40         & 0.02949   & 40         & 0.029409   \\ 
        4  & 25         & 0.02905414 & 25         & 0.02905   & 25         & 0.029237   \\ 
        5  & 5          & 0.02845183 & 5          & 0.02845   & 19         & 0.028443   \\ 
        6  & 20         & 0.02821648 & 20         & 0.02822   & 5          & 0.028404   \\ 
        7  & 19         & 0.02821330 & 19         & 0.02821   & 20         & 0.028175   \\ 
        8  & 18         & 0.02811892 & 18         & 0.02812   & 18         & 0.028076   \\ 
        9  & 16         & 0.02674112 & 16         & 0.02674   & 28         & 0.026648   \\ 
        10 & 28         & 0.02669419 & 28         & 0.02669   & 16         & 0.026637   \\ 
        11 & 15         & 0.02658757 & 15         & 0.02659   & 15         & 0.026344   \\ 
        12 & 12         & 0.02636156 & 12         & 0.02636   & 12         & 0.026183   \\ 
        13 & 21         & 0.02627735 & 21         & 0.02628   & 21         & 0.026183   \\ 
        14 & 11         & 0.02602722 & 11         & 0.02603   & 11         & 0.026039   \\ 
    \end{tabular}
    \caption{The $ 14 $ most-frequently visited states in the $ 42 $
    state model with Jail states $ 10 $, $ 41 $ and $ 42 $
    combined.}~%
    \label{tab:monopoly:freq42}
\end{table}

\begin{figure}
    \centering
    \includegraphics[scale=0.50]{monopoly42}
    \caption{Steady state distribution for the expanded $ 42 $-state
    model of Monopoly with Jail states $ 10 $, $ 41 $ and $ 42 $
    combined.}~%
    \label{fig:monopoly:monopoly42}
\end{figure}

The specific order of most frequently visited states is not as important
as it appears in Table~%
\ref{tab:monopoly:freq42}, since differences in frequency generally
appear in the fourth significant digit.  More useful is to consider
\emph{sets of most frequently visited states}.  The most frequently
visited set of states is \( 10 \), \( 41 \), \( 42 \), \( 24 \) and \(
40 \).

There is still one more detailed rule to consider, any player who rolls
three doubles in a row goes to Jail.  One way to handle this detail is
to add \( 80 \) new states to the model as before, keeping track of
history in a model with a total of \( 122 \) states.  Again this
illustrates the principle of turning a process which is not Markov into
a Markov chain by adding states.  An alternative way to handle this is
to make an approximation by adjusting the Jail matrix \( J \) so that it
sends a player not already in Jail to Jail with probability \( 1/216 \)
since the probability of rolling three doubles in a row is \( \frac{1}{216}
\).  Replace each \( 1 \) on the diagonal with \( \frac{215}{216} \) and
add \( \frac{1}{216} \) to column \( 41 \). After making these last
adjustments to the probability transition matrix, Table~%
\ref{tab:monopoly:finalmonopoly} summarizes the results of simulation.
The steady state distribution for this final Markov chain model as a bar
graph is in Figure~%
\ref{fig:monopoly:finalmonopoly}.

\begin{table}
    \centering
    \begin{tabular}{rrr}
           & State      & Frequency   \\ 
        1  & 10         & 0.117841272 \\ 
        2  & 24         & 0.029795914 \\ 
        3  & 40         & 0.029055135 \\ 
        4  & 18         & 0.028235759 \\ 
        5  & 20         & 0.028148582 \\ 
        6  & 19         & 0.028019441 \\ 
        7  & 25         & 0.027108791 \\ 
        8  & 16         & 0.026894000 \\ 
        9  & 5          & 0.026694388 \\ 
        10 & 28         & 0.026549407 \\ 
        11 & 12         & 0.026091348 \\ 
        12 & 21         & 0.025963983 \\ 
        13 & 11         & 0.025481659 \\ 
        14 & 23         & 0.025471064 \\ 
        15 & 26         & 0.025384224 \\ 
        16 & 27         & 0.025192088 \\ 
        17 & 31         & 0.025076589 \\ 
        18 & 15         & 0.025064206 \\ 
        19 & 39         & 0.024766786 \\ 
        20 & 32         & 0.024499247 \\ 
        21 & 29         & 0.024398944 \\ 
        22 & 14         & 0.024354843 \\ 
        23 & 34         & 0.023334506 \\ 
        24 & 17         & 0.022984619 \\ 
        25 & 35         & 0.022713579 \\ 
        26 & 33         & 0.022086039 \\ 
        27 & 4          & 0.021986015 \\ 
        28 & 8          & 0.021786854 \\ 
        29 & 13         & 0.021756287 \\ 
        30 & 9          & 0.021606785 \\ 
        31 & 6          & 0.021335917 \\ 
        32 & 37         & 0.020478819 \\ 
        33 & 3          & 0.020476878 \\ 
        34 & 38         & 0.020438135 \\ 
        35 & 1          & 0.020087639 \\ 
        36 & 2          & 0.017758354 \\ 
        37 & 22         & 0.012115464 \\ 
        38 & 7          & 0.009514394 \\ 
        39 & 36         & 0.009452046 \\ 
        40 & 30         & 0.000000000 \\ 
    \end{tabular}
    \caption{The frequency of visited states in the final $ 42 $-state
    model with the approximation for going to Jail by rolling three
    doubles.}~%
    \label{tab:monopoly:finalmonopoly}
\end{table}

\begin{figure}
    \centering
    \includegraphics[scale=0.50]{finalmonopoly}
    \caption{Steady state distribution for the final $ 42 $-state
    model of Monopoly.}~%
    \label{fig:monopoly:finalmonopoly}
\end{figure}

\subsection*{Additional Modeling Considerations} A few modeling
assumptions are not explicitly considered in this model.  One assumption
is that Community Chest and Chance card decks are shuffled after each
draw so that drawing the cards is a random event, possibly even
repeating.  The game actually has \( 17 \) Community Chest cards
including a ``Get Out of Jail Free'' card, which can be saved by the
player who draws it.  Assuming that card has been drawn and retained
gives the denominator of \( 16 \) used in the probabilities of the
Community Chest matrix.  Likewise, Monopoly uses \( 16 \) Chance cards
but one is another ``Get Out of Jail Free'' card which can be saved.
Assuming that card is still in the Chance deck gives the denominator of \(
16 \) used in the probabilities of the Chance matrix.  This pair of
mixed assumptions makes uniform denominators in the two matrices.  Also,
the final model assumes that appearance of a third consecutive roll of
doubles is equally likely to occur at any space on the board, which is
not strictly correct.  Some squares are already known to be visited more
than others and a third consecutive doubles could occur at an even
number of squares further on the board from where the doubles run began.

In~%
\cite{abbott97}, Abbott and Richey use the steady state distribution to
find the expected rental value and break-even times of the various color
group properties.  However, since illustration of the modeling and
simulation of large Markov chains is the focus of these notes, these
notes do not follow that investigation.

Many sites on the Internet make essentially the same Markov chain
analysis of Monopoly in varying degrees of detail and in slightly
different ways.  For example, some use a \( 120 \)-state chain to model
the ``rolling doubles'' rule, and then condense the chain to the \( 40 \)
board states.  All reach essentially the same conclusions.  Many of
those sites go further and make an expected Return on Investment
analysis to determine the most valuable spaces on the board, not just
the most frequently visited.  The Return on Investment analysis is not
pursued here because the intention is to illustrate the Markov chain
aspects.

\visual{Section Starter Question}{../../../../CommonInformation/Lessons/question_mark.png}
\section*{Section Ending Answer}

Since the board has \( 40 \) spaces, a first guess is that a token
spends about \( 1/40 = 0.025 \) of the time on Boardwalk.  In fact,
using the most detailed Markov chain here, \( 0.02431 \) of the time is
spent at Boardwalk.  It is less than the easy estimate because the model
shows that a much larger fraction of time is spent in Jail, and in fact,
not all spaces are equally likely to be occupied.

\subsection*{Sources}

The subsection on Monopoly as a Markov chain is based on~%
\cite{abbott97} with more information from~%
\cite{ash72}.  Matthew Sheby pointed out an error in the Chance
probability matrix in the scripts, now corrected so the steady state
calculations agree with
\cite{abbott97}.  The simulation script is adapted from \booktitle{Efficient
R}, Chapter 7 by Colin Gillespie and Robin Lovelace.

\hr

\visual{Algorithms, Scripts, Simulations}{../../../../CommonInformation/Lessons/computer.png}
\section*{Algorithms, Scripts, Simulations}

\subsection*{Algorithm}
\begin{codebox}
  \Procname{Monopoly Game Markov Chain}
  \zi Comment Post:  Steady state distribution
  \zi of Markov chains modeling the Monopoly game
  \li Load Markov chain library
  \li \proc{rotvec}
  \li Cyclically rotate vector elements one place to right
  \zi
  \li Set number of states and state names
  \zi
  \li Fill diceRoll matrix
  \zi
  \li Fill diagonal Jail matrix with move from Police to Jail
  \zi
  \li Fill diagonal Community Chest matrix with moves to other states
  \zi
  \li Fill diagonal Chance matrix with moves to other states
  \zi
  \li Transition probability matrix as product of
  \zi diceRoll, Jail, Chance and Community Chest
  \li Create Monopoly Markov chain object
  \li Find steady state distribution of Markov Chains
  \li Report selected results in table or graph 
\end{codebox}

\begin{codebox}
  \Procname{Simulate Monopoly Game}
  \zi Comment Post: Empirical stationary distributions
  \zi from simulation of the Monopoly game
  \li \proc{moveSquare}
  \li Simulate dice roll and
  \li keep track of jail time
  \li return new state and jail time
  \zi
  \li \proc{checkState}
  \li Cyclically reduce state about state 40,
  \li return state number
  \zi
  \li \proc{updateStateVector}
  \li Keep track of number of times a state is visited
  \li return updated $\id{landings}$
  \zi
  \li \proc{communityChest}
  \li Move state according to random selection
  \li of Community Chest card
  \li return new state and jail time
  \zi
  \li \proc{Chance}
  \li Move state according to random selection
  \li of Chance card,
  \li return new state and jail time
  \zi
  \li \proc{mySimulateMonopoly}
  \li Initialize $\id{landings}$ vector to hold occupancy of states
  \li Initialize $\id{jailTime}$
  \li Initialize first state at Go
  \li \For $i = 1$ \To $\id{numberOfTurns}$
  \li         Move by \id{moveSquare}
  \li         \If on Chance square, move by \proc{chance}
  \li         \If on Community Chest square, move by \proc{communityChest}
  \li            Update  $\id{landings}$
\end{codebox}
\subsection*{Scripts}


\begin{description}

% \item[Geogebra] 

% \link{  .ggb}{GeoGebra applet}

\item[R] 

\link{http://www.math.unl.edu/~sdunbar1/monopoly40.R}{R script for
  40-state Monopoly.}

\begin{lstlisting}[language=R]
library("markovchain")

rotvec <- function(vec) vec[ c(length(vec), 1:(length(vec)-1)) ]

nStates <- 40
monopolyStates <- as.character( c(1:nStates) )

diceRoll <- matrix(0, nStates, nStates)
p1 <- c(0, 0,
        1/36, 2/36, 3/36, 4/36, 5/36, 6/36, 5/36, 4/36, 3/36, 2/36, 1/36,
        numeric(nStates-13)
        )
for (i in 1:nStates) {
    diceRoll[i, ] <- p1
    p1 <- rotvec(p1)
}

Jail <- diag(nStates)
Jail[30,30] <- 0
Jail[30, 10] <- 1

Ch <- diag(nStates)
chanceStates <- c(7, 22, 36)
chanceDests <- c(5, 10, 11, 24, 39, 40)
Ch[ cbind( chanceStates, chanceStates ) ] <- 7/16
Ch[ cbind( chanceStates, chanceStates-3 ) ] <- 1/16 #Go Back 3 spaces
Ch[ chanceStates, chanceDests] <- 1/16
Ch[7,12] <- 1/16; Ch[22,28] <- 1/16; Ch[36, 12] <- 1/16 #nearest utility
Ch[7,15] <- 1/16; Ch[22,25] <- 1/16; #nearest Railroad,
Ch[36, 5] <- 2/16 # note 2 ways to get to Reading Railroad

Cc <- diag(nStates)
chestStates <- c(2, 17, 33)
chestDests <-  c(10, 40)
Cc[ cbind(chestStates, chestStates) ] <- 14/16
Cc[chestStates, chestDests] <- 1/16

monopolyMatrix <- diceRoll %*% Jail %*% Ch %*% Cc

mcMonopoly <- new("markovchain", states = monopolyStates, byrow = TRUE,
                  transitionMatrix = monopolyMatrix, name = "Monopoly")
ss <- steadyStates(mcMonopoly)

top14ofAbbott <- (1/100)*c(5.896, 3.188, 3.114, 3.071, 3.064, 2.993, 2.919, 2.917, 2.875, 2.830, 2.811, 2.777, 2.739, 2.736)
top14ofAbbottIndices <- c(10, 24, 40, 19, 25, 5, 15, 18, 20, 21, 28, 16, 23, 11)

sstop14 <- sort(ss, decreasing=TRUE)[1:14]
sstop14Indices <- order(ss, decreasing=TRUE)[1:14]

maxDiff <- max( abs( sstop14 - top14ofAbbott) )
cat("Maximum Difference: ", maxDiff, "\n")
maxDiffIndex <- which.max( abs( sstop14 - top14ofAbbott) )
cat("Maximum Difference Index: ", maxDiffIndex, "\n")
maxRelDiff <- maxDiff/sstop14[maxDiffIndex]
cat("Relative Maximum Difference: ", maxRelDiff, "\n")

ssdf <- data.frame( sstop14Indices, sstop14)

barplot(ss, names.arg=1:40,
        main="Steady State Distribution for 40-State Monopoly",
        xlab="State", ylab="Probability")
\end{lstlisting}

\link{http://www.math.unl.edu/~sdunbar1/monopoly42.R}{R script for
  42-state Monopoly.}

\begin{lstlisting}
  library("markovchain")

rotvec <- function(vec) vec[ c(length(vec), 1:(length(vec)-1)) ]

nStates <- 42
monopolyStates <- as.character( c(1:nStates) )

diceRoll <- matrix(0, (nStates-2), (nStates-2))
p1 <- c(0, 0,
        1/36, 2/36, 3/36, 4/36, 5/36, 6/36, 5/36, 4/36, 3/36, 2/36, 1/36,
        numeric((nStates-2)-13)
        )
for (i in 1:(nStates-2)) {                #regular states 1 through 40
    diceRoll[i, ] <- p1
    p1 <- rotvec(p1)
}
## Next, no probability of going from board state to extra Jail states 
diceRoll  <- cbind(diceRoll, matrix(0, (nStates-2), 2))
## add rows from extra Jail states
diceRoll  <- rbind(diceRoll, matrix(0, 2, nStates))
## fill transitions for rolling doubles to get out of Jail
diceRoll[41, 41] <- 0; diceRoll[41, 42] <- 5/6 #stay in jail one turn
diceRoll[41, c(12, 14, 16, 18, 20, 22)] <- 1/36 #roll doubles
diceRoll[42, 42] <- 0; diceRoll[42, 10] <- 5/6 #stay in Jail one more turn
diceRoll[42, c(12, 14, 16, 18, 20, 22)] <- 1/36 #roll doubles 

Jail <- diag(nStates)
Jail[30,30] <- 0
Jail[30, 41] <- 1                       #go from Police to Jail

Cc <- diag(nStates)
chestStates <- c(2, 17, 33)
chestDests <-  c(40, 41)
Cc[ cbind(chestStates, chestStates) ] <- 14/16
Cc[chestStates, chestDests] <- 1/16

Ch <- diag(nStates)
chanceStates <- c(7, 22, 36)
chanceDests <- c(5, 11, 24, 39, 40, 41)
Ch[ cbind( chanceStates, chanceStates ) ] <- 7/16
Ch[ cbind( chanceStates, chanceStates-3 ) ] <- 1/16 #Go Back 3 spaces
Ch[ chanceStates, chanceDests] <- 1/16
Ch[7,12] <- 1/16; Ch[22,28] <- 1/16; Ch[36, 12] <- 1/16 #nearest utility
Ch[7,15] <- 1/16; Ch[22,25] <- 1/16; #nearest Railroad,
Ch[36, 5] <- 2/16 # note 2 ways to get to Reading Railroad

monopMat <- diceRoll %*% Jail %*% Ch %*% Cc

mcMonopoly42 <- new("markovchain", states = monopolyStates, byrow = TRUE,
                  transitionMatrix = monopMat, name = "Monopoly")
ss <- steadyStates(mcMonopoly42)

ssComb  <- ss
ssComb[10]  <- ssComb[10] + ssComb[41] + ssComb[42]
ssCombTop14 <- sort(ssComb[1:40], decreasing=TRUE)[1:14]
ssCombTop14Indices <- order(ssComb[1:40], decreasing=TRUE)[1:14]
ssCombdf <- data.frame( ssCombTop14Indices, ssCombTop14)

top14ofAbbott <- (1/100)*c(10.800, 3.013, 2.949, 2.905, 2.845, 2.822, 2.821, 2.674, 2.669, 2.659, 2.659, 2.636, 2.628, 2.603)
top14ofAbbottIndices <- c(10, 24, 40, 25, 5, 20, 19, 18, 16, 28, 15, 12, 21 , 11)

maxDiff <- max( abs( ssCombTop14 - top14ofAbbott) )
cat("Maximum Difference: ", maxDiff, "\n")
maxDiffIndex <- which.max( abs( ssCombTop14 - top14ofAbbott) )
cat("Maximum Difference Index: ", maxDiffIndex, "\n")
maxRelDiff <- maxDiff/ssCombTop14[maxDiffIndex]
cat("Relative Maximum Difference: ", maxRelDiff, "\n")

barplot(ssComb[1:40], names.arg=1:40,
        main="Steady State Distribution for 42-State Monopoly, 10,41,42 Combined",
        xlab="State", ylab="Probability")
\end{lstlisting}

\link{http://www.math.unl.edu/~sdunbar1/simulateMonopoly42.R}{R script to
  simulate 42-state Monopoly.}

\begin{lstlisting}
  ##  This function calculates where you will land on your next move.
moveSquare <- function(current, jailTime) {
    dice <- sample(seq(1,6), 2, replace=TRUE)

    if (current == 10 & jailTime > 0) { # in Jail
        if (dice[1] == dice[2]) {           # roll doubles!
            current <- 10 + sum(dice)       # get out of Jail
            jailTime <- 0
        } else {                            #don't move but...
            jailTime <- (jailTime + 1) %% 3 # count up jail time
        } 
    } else {
        current <- current + sum(dice)      # regular roll
    }
  return(c(current, jailTime))
}

## Helper function to avoid code replication
checkState <- function(current) {
  if (current > 40) {
    current <- current - 40
  } else if (current < 1) {
    current <- current + 40
  }
  return(current)
}

## Helper function to record landings, 
updateStateVector <- function(current, jailTime, landings) {
    if (current == 10 & jailTime > 0) {
        landings[40 + jailTime] = landings[40 + jailTime] + 1
    } else {
        landings[current] <- landings[current] + 1
    }
    
    return(landings)
}

## Move according to community chest cards
## Use 16 cards, Get Out of Jail is reserved
communityChest <- function(current, jailTime) {
  u <- runif(1)
  goto <- current # Default. Do nothing
  if (u < 1 / 16) {
    goto <- 40 # Go
  } else if (u < 2 / 16) {
      goto <- 10 # Jail
      jailTime <- 1
  }
  return( c(goto, jailTime) )
}

## Move according to chance cards
## Use 16 cards, Get Out of Jail is NOT reserved
chance <- function(current, jailTime) {
  u <- runif(1)
  goto <- current # Default. Do nothing
  if (u < 1 / 16) {
      goto <- 40 # Go
  } else if (u < 2 / 16) {
      goto <- 24 # Illinois Ave.
  } else if (u < 3 / 16) {
      goto <- 11 # St. Charles Place
  } else if (u < 4 / 16) {
      goto <- 10 # Jail
      jailTime <- 1
  } else if (u < 5 /16 ) {
      goto <- 5 #Take ride on Reading
  } else if (u < 6 / 16) {
      goto <- 39 # Boardwalk
  } else if (u < 7 / 16) {
      goto <- checkState(current - 3) # Must check, since goto maybe negative!
  } else if (u < 8 / 16) {
      if (current > 28  | current < 12) { # Advance to nearest Utility
      goto <- 12 # Water Works
    } else {
      goto <- 28 # Electricity
    }
  } else if (u < 9 / 16) { # Advance to nearest Railroad
    if (current > 35  | current < 5) { 
      goto <- 5                         # Reading
    } else if (current > 5  & current < 15) {
      goto <- 15 #Pennsylvania
    } else if (current > 15  & current < 25) {
      goto <- 25 # B & O
    } else if (current > 25  & current < 35) {
      goto <- 35 # Short Line
    }
}
  return(c(goto, jailTime))
}

## Main, record where landings occur
mySimulateMonopoly <- function (numberOfTurns) 
{
    landings <- numeric(42) # vector to hold occupancy of states,
    # 1 is Med Ave, 40 is Go
    # 41 is in Jail for one turn, 42 is Jail for two turns 
    # 10 is Just Visiting Jail or Jail for three turns, ready to get out
    jailTime <- 0
    current <- 40           # start at Go
    for (i in 1:numberOfTurns) {
        moveState <- moveSquare(current, jailTime)
        current <- moveState[1]
        jailTime <- moveState[2]
        current <- checkState(current)
        if (current == 30) {
            current <- 10
            jailTime <- 1
        }
        if (current == 7 | current == 22 | current == 36) {
            move <- chance(current, jailTime)
            current <- move[1]
            jailTime <- move[2]
        }
        if (current == 2 | current == 17 | current == 33) {
            move <- communityChest(current, jailTime)
            current <- move[1]
            jailTime <- move[2]
        }
        landings <- updateStateVector(current, jailTime, landings) # must do update
    }
return(landings)
}

n  <- 1000
ss  <- mySimulateMonopoly(n)/n
\end{lstlisting}
%\item[Octave]

% \link{http://www.math.unl.edu/~sdunbar1/    .m}{Octave script for .}

% \begin{lstlisting}[language=Octave]

% \end{lstlisting}

% \item[Perl] 

% \link{http://www.math.unl.edu/~sdunbar1/    .pl}{Perl PDL script for .}

% \begin{lstlisting}[language=Perl]

% \end{lstlisting}

% \item[SciPy] 

% \link{http://www.math.unl.edu/~sdunbar1/    .py}{Scientific Python script for .}

% \begin{lstlisting}[language=Python]

% \end{lstlisting}

\end{description}



%%% Local Variables:
%%% mode: latex
%%% TeX-master: t
%%% End:


\hr

\visual{Problems to Work}{../../../../CommonInformation/Lessons/solveproblems.png}
\section*{Problems to Work for Understanding}
\renewcommand{\theexerciseseries}{}
\renewcommand{\theexercise}{\arabic{exercise}}

\begin{exercise}
    Create a small Monopoly game with just \( 8 \) spaces:
    \begin{enumerate}
        \item
            Reading Railroad is state \( 1 \),
        \item
            Jail is state \( 2 \),
        \item
            Pennsylvania Railroad is state \( 3 \),
        \item
            Policeman is state \( 4 \),
        \item
            B.\ \& O.\ Railroad is state \( 5 \),
        \item
            Chance is state \( 6 \),
        \item
            Short Line is state \( 7 \) and
        \item
            Go is state \( 8 \).
    \end{enumerate}
    A fair two-sided ``die'' (i.e.\ a fair coin) determines the \( 1 \)
    or \( 2 \) step forward motion of a player token around the board.
    Considering this as a Markov Chain, find the stationary
    distribution.  Assume the simple rule that a player leaves Jail
    immediately, no staying for \( 2 \) more turns. Be sure to include
    the effect drawing an ``Advance to the Nearest Railroad'', ``Take a
    Ride on the Reading'', ``Go to Jail'', ``Advance to Go'', or ``Go
    back \( 3 \) spaces'' card from the Chance deck.  All other Chance
    movement cards (e.g. Illinois Avenue, Boardwalk, Utility) have no
    effect on movement.
\end{exercise}
\begin{solution}
    Use the following script:
\begin{lstlisting}
    library("markovchain")

rotvec <- function(vec) vec[ c(length(vec), 1:(length(vec)-1)) ]

nStates <- 8
monopolyStates <- as.character( c(1:nStates) )

diceRoll <- matrix(0, nStates, nStates)
p1 <- c(0, 1/2, 1/2, 0, 0, 0, 0, 0)

for (i in 1:nStates) {
    diceRoll[i, ] <- p1
    p1 <- rotvec(p1)
}

Jail <- diag(nStates)
Jail[4, 4] <- 0
Jail[4, 2] <- 1                       #go from Police to Jail

Ch <- diag(nStates)
Ch[6, 7] <- 1/16                         #Nearest Railroad
Ch[6, 8] <- 1/16                         #Advance to Go
Ch[6, 1] <- 1/16                         #Reading Railroad
Ch[6, 2] <- 1/16                         #Go to Jail
Ch[6, 3] <- 1/16                         #Go back 3
Ch[6, 6] <- 11/16                         #All other cards

monopMat <- diceRoll %*% Jail %*% Ch

mcMonopoly8 <- new("markovchain", states = monopolyStates, byrow = TRUE,
                  transitionMatrix = monopMat, name = "Monopoly")
ss <- steadyStates(mcMonopoly8)
cat(ss, "\n")
\end{lstlisting}

    The steady state distribution is \begin{tabular}{rr}
        1  & 0.07731016 \\ 
        2  & 0.383806   \\ 
        3  & 0.2342177  \\ 
        4  & 0.0        \\ 
        5  & 0.1171089  \\ 
        6  & 0.04025618 \\ 
        7  & 0.08234218 \\ 
        8  & 0.06495883 \\ 
    \end{tabular}
\end{solution}

\hr

\visual{Books}{../../../../CommonInformation/Lessons/books.png}
\section*{Reading Suggestion:}

\bibliography{../../../../CommonInformation/bibliography}

%   \begin{enumerate}
%     \item
%     \item
%     \item
%   \end{enumerate}

\hr

\visual{Links}{../../../../CommonInformation/Lessons/chainlink.png}
\section*{Outside Readings and Links:}
\begin{enumerate}
    \item
    \item
    \item
    \item
\end{enumerate}

\section*{\solutionsname} \loadSolutions

\hr

\mydisclaim \myfooter

Last modified:  \flastmod

\end{document}

File name                  : monopoly.tex
Number of characters       : 29150
Number of words            : 3318
Percent of complex words   : 18.08
Average syllables per word : 1.7384
Number of sentences        : 247
Average words per sentence : 13.4332
Number of text lines       : 625
Number of blank lines      : 93
Number of paragraphs       : 92


READABILITY INDICES

Fog                        : 12.6066
Flesch                     : 46.1319
Flesch-Kincaid             : 10.1620


